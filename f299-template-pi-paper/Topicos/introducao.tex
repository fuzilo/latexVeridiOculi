Atualmente existem hoje no mundo 58.497 espécies de árvores, sendo que a região da américa latina compreende a distribuição de 23.631 espécies.\cite{bgci}

Contudo, entre os anos de 2020 e 2021 foram desflorestados 21.642 ha. da Mata Atlântica brasileira, equivalente a mais de 28000 campos de futebol. Os dados foram apresentados no Atlas dos Remanescentes Florestais da Mata Atlântica, uma colaboração entre o Instituto de Pesquisas Espaciais(INPE) e a Fundação SOS Mata Atlântica. O Relatório Anual de 2022 também indica que este foi o valor foi o mais alto desde 2015 e 90\% maior do que o menor valor da história, alcançado em 2018.\cite{atlas-mata-atlantica}

Segundo um levantamento feito  em 2020 pelo MapBiomas, uma rede Colaborativa que mapeia anualmente a cobertura e uso do solo,apenas 29,7\% do total do bioma Mata Atlântica tem formação Florestal.\cite{mapbiomas}

O Ministério do Meio Ambiente, baseando-se na classificação de risco de extinção do BGCI, por intermédio da portaria Nº 43 de 31 de janeiro de 2014 \cite{mma2014},  Institui o Programa Nacional de Conservação das Espécies Ameaçadas de Extinção - Pró-Espécies. A portaria Nº 443 de 17 de dezembro de 2014\cite{mma20142}, atualiza a lista de espécies de plantas em risco de extinção, atualizada novamente pela portaria Nº 148 de 7 de junho de 2022.\cite{mma2022}

Associado ao desmatamento está a perda da biodiversidade das espécies de plantas. Dados do Botanic Gardens Conservation International(BGCI), demonstram que 30\% das espécies de árvores da américa latina encontram-se ameaçadas de extinção.\cite{bgci}

Com o intuito de guiar e coordenar o desenvolvimento sustentável no mundo, as Organizações das Nações Unidas idealizaram os 17 Objetivos de Desenvolvimento Sustentável, para erradicação da pobreza, proteção do meio ambiente e clima. Dentre esses objetivos, destacamos a ODS 15 que visa proteger, restaurar e promover o uso sustentável de ecossistemas terrestres bem como gerenciar florestas de forma sustentável. Assim, proteger as florestas nativas e ecossistemas de serem degradados impactam diretamente nas ações Globais contra a Mudança do Clima, objeto da ODS 13. Ainda, segundo o relatório MapBiomas, 70\% da Mata Atlântica tem uso antrópico sendo esta a prioridade da ODS 11, que busca tornar assentamentos humanos seguros e sustentáveis.\cite{agenda2030}

No decorrer deste estudo será realizada a identificação de uma espécie arbórea de nome científico \textit{Cecropia pachystachya}, popularmente conhecida como Embaúba. Segundo Carvalho (2006) em seu manual de Espécies Arbóreas, a Embaúba é uma árvore perenifólia que atinge cerca de 25 metros de altura e 45 centímetros de DAP (diâmetro à altura do peito, medido a 1,30m do solo) em sua fase adulta. Seu tronco é reto, cilíndrico e fistuloso (oco por dentro). A casca externa é áspera e cinza-clara, com até 6mm de espessura, enquanto a casca interna é alaranjada-rosada e fibrosa. Suas folhas são alternadas e agrupadas nas extremidades dos ramos, com lâmina de 20 a 35 cm de comprimento por 20 a 35 cm de largura. Elas são palmatilobadas e divididas em 5 a 12 lobos desiguais obovados, separados até o pecíolo por espaços de 2 a 3 cm, e densamente esbranquiçadas-tomentosas na face inferior. A face superior apresenta pelos curtos e esparsos, margem inteira ou ligeiramente ondulada e ápice obtuso, com nervura central proeminente na face inferior. O pecíolo é forte e comprido, medindo de 16 a 25 cm de comprimento, com pelos uncinados e cálix na base. É uma espécie dióica, ou seja, possui indivíduos machos e fêmeas.\cite{carvalho2006embauba}
A Embaúba é uma espécie pioneira, associada a capoeiras novas, situadas junto a vertentes ou cursos d'água, estabelecendo-se rapidamente em clareiras descampadas. Popularmente é conhecida por seu uso medicinal da folha e casca. \cite{carvalho2006embauba}

Andrade et. al avaliaram o potencial antimicrobiano e antifúngico do extrato etanólico de folhas da Embaúba, que demonstrou atividade contra linhagens bacterianas \textit {Staphylococcus aureus}, \textit {Streptococcus pyrogenes} e a levedura Candida albicans levando a resultados promissores.\cite{de2021avaliaccao}

No presente trabalho, será usada a classificação de risco, desenvolvido pela Lista Vermelha da IUCN, um sistema destinado a classificar espécies em alto risco de extinção global e de forma amplamente compreensível. Utiliza procedimentos de avaliação padronizados para atribuir espécies a diferentes categorias de risco de extinção com base em cinco critérios quantitativos, incluindo medidas de tamanho populacional, restrição de distribuição geográfica e taxa de declínio.  (IUCN Red List, 2021)
No Brasil, a classificação da lista Vermelha é adotada pelo Ministério do Meio Ambiente através da portaria Nº 148 de 7 de junho de 2022, que lista as espécies de plantas em risco de extinção dos biomas brasileiros. (MMA, 2022)

A partir dessa classificação será possível classificar espécies quanto o seu grau de risco de extinção e, num segundo momento, propor ações de manejo sustentável a fim de rastrear as espécies de árvores alvo da pesquisa, para esse fim será utilizado o sistema de coordenadas SIRGAS.\par

O SIRGAS (Sistema de Referência Geodésico para as Américas) é um sistema de referência geodésico associadas a uma determinada época de referência e a sua variação ao longo do tempo é tomada em consideração, seja pelas velocidades individuais das estações, ou por um modelo de velocidade contínuo que compreende os movimentos das placas litosféricas e as deformações na crosta. \cite{SIRGAS2023}

Para a coleta de dados em campo, foi empregada a escala de Likert como método de obtenção de informações. Essa técnica é reconhecida e frequentemente utilizada em pesquisas científicas, permitindo que os participantes expressem suas opiniões, atitudes ou experiências em relação a variáveis específicas de interesse. A escala de Likert consiste em uma série de afirmações ou questões nas quais os respondentes atribuem um grau de concordância ou discordância, geralmente em uma escala que varia de "discordo totalmente" a "concordo totalmente", com diferentes níveis intermediários.  Segundo Nogueira (2002), essa abordagem oferece uma maneira estruturada e quantificável de coletar dados sobre uma ampla variedade de fenômenos, fornecendo insights valiosos para análises posteriores. Essa escala pode ser adaptada com diferentes quantidades de pontos ou níveis de detalhamento, visto que um número maior de pontos proporciona respostas mais precisas.

De forma  contribuir com gestão dos recursos florestais de forma a reduzir o desmatamento e garantir a preservação das espécies de árvores brasileiras, o presente trabalho tem como objetivo desenvolver um Sistema Web para identificação de flora por imageamento aéreo utilizando Inteligência Artificial (IA), capaz de identificar árvores de interesse econômico e ambiental, importantes para equilíbrio da biodiversidade, bem como espécies invasoras, que podem implicar na perda da biodiversidade, além de identificar espécies em risco de extinção. A aplicação destina-se à elaboração de inventários florestais, em projetos de empreendimentos imobiliários, bem como embasamento técnico para propostas de manejo e rastreamento de árvores na região da mata Atlântica.