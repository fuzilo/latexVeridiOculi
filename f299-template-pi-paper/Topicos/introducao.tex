Estudos indicam que a América Latina abriga 23.631 das 58.497 espécies de árvores existentes no mundo \cite{bgci}. Entretanto, dados recentes mostram que o desmatamento na Mata Atlântica brasileira atingiu 21.642 hectares entre 2020 e 2021 - área equivalente a mais de 28 mil campos de futebol, esse valor representa o maior índice registrado desde 2015 \cite{atlas-mata-atlantica}. Apenas 29,7\% da área total desse bioma ainda mantém cobertura florestal original, evidenciando seu elevado grau de degradação \cite{mapbiomas}. O desmatamento está diretamente associado à perda de biodiversidade vegetal, cerca de 30\% das espécies de árvores na América Latina estão ameaçadas de extinção, evidenciando os impactos da degradação ambiental. Diante desse cenário, a Organização das Nações Unidas (ONU) estabeleceu os 17 Objetivos de Desenvolvimento Sustentável (ODS), com metas globais para erradicar a pobreza, proteger o meio ambiente e combater as mudanças climáticas. Nesse contexto, destaca-se a ODS 15, que tem como objetivo proteger, restaurar e promover o uso sustentável dos ecossistemas terrestres, além de gerir as florestas de forma equilibrada, visando frear a perda de biodiversidade. A proteção de florestas nativas e ecossistemas impacta diretamente nas ações globais contra as mudanças climáticas, alinhando-se ao ODS 13 (Ação Contra a Mudança Global do Clima). Além disso, dados do MapBiomas revelam que 70\% da Mata Atlântica já sofreu alteração antrópica, o que reforça a urgência do ODS 11 (Cidades e Comunidades Sustentáveis), que visa promover assentamentos humanos mais seguros e ambientalmente equilibrados \cite{agenda2030}. 

No decorrer deste estudo foi realizada a identificação de uma espécie arbórea de nome científico \textit{Cecropia pachystachya}, popularmente conhecida como Embaúba, uma árvore perenifólia que atinge cerca de 25 metros de altura e 45 centímetros de diâmetro à Altura do Peito (DAP) em sua fase adulta. Seu tronco é reto, cilíndrico e fistuloso (oco por dentro). A casca externa é áspera e cinza-clara, com até 6mm de espessura, enquanto a casca interna é alaranjada-rosada e fibrosa. Suas folhas são alternadas e agrupadas nas extremidades dos ramos, com lâmina de 20 a 35 cm de comprimento por 20 a 35 cm de largura. Elas são palmatilobadas e divididas em 5 a 12 lobos desiguais obovados, separados até o pecíolo por espaços de 2 a 3 cm, e densamente esbranquiçadas-tomentosas (com penugem) na face inferior. A face superior apresenta pêlos curtos e esparsos, margem inteira ou ligeiramente ondulada e ápice obtuso, com nervura central proeminente na face inferior. O pecíolo é forte e comprido, medindo de 16 a 25 cm de comprimento, com pelos uncinados e cálix na base. É uma espécie dióica, ou seja, possui indivíduos machos e fêmeas. A Embaúba é uma espécie pioneira, associada a capoeiras novas, áreas de vegetação secundária que surgem em terrenos previamente desmatados, situadas junto a vertentes ou cursos d’água, estabelecendo-se rapidamente em clareiras descampadas. Popularmente é conhecida por seu uso medicinal da folha e casca. \cite{carvalho2006embauba} 

Em uma pesquisa de campo conduzida pelos autores, realizada por meio de questionários aplicados em 3 instituições (CETESB, Fundação Florestal e um escritório de Engenharia Ambiental), avaliou a adoção de tecnologias para preservação ambiental. Todos os entrevistados confirmaram compromisso com a causa, mas 78\% destacaram o alto custo de levantamentos de flora como uma barreira para garantir a preservação. Embora a maioria utilize tecnologias como drones, GIS e imagens aéreas, 44\% desconfiam de análises por Inteligência Artificial (IA), enquanto 56\% veem potencial como ferramenta auxiliar. A carência de recursos atualizados (como aerofotos recentes) eleva custos e limita inventários florestais, refletindo a divisão de opiniões sobre a viabilidade da IA na tomada de decisões.

Este trabalho visa desenvolver uma aplicação baseada em Inteligência Artificial (IA) para a identificação da Cecropia pachystachya (embaúba) por meio de imagens aéreas. A espécie, de relevância econômica e ambiental, desempenha um papel crucial na manutenção da biodiversidade. A ferramenta proposta tem como finalidade auxiliar na elaboração de inventários florestais e no levantamento detalhado de espécies arbóreas, servindo de suporte para projetos de manejo, conservação e empreendimentos imobiliários na Mata Atlântica. Além disso, busca fornecer embasamento técnico para propostas de rastreamento e preservação desta espécie emblemática.