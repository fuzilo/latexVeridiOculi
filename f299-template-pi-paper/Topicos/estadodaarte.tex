A identificação de árvores de interesse ambiental e econômico torna-se um desafio em grandes áreas, especialmente devido à dificuldade de distinguir entre espécies com características semelhantes. A utilização de aprendizado profundo e Inteligência Artificial pode otimizar esse processo, tornando-o mais rápido, eficiente e com alta acurácia. Essa abordagem tecnológica possibilita a geração de dados confiáveis para subsidiar tomadas de decisão e a elaboração de estratégias de manejo sustentável.
Para classificar e mapear plantas ameaçadas na Mata Atlântica de forma colaborativa, de Souza et al. (2020) desenvolveram um aplicativo baseado em uma arquitetura MobileNet adaptada. Os autores utilizaram imagens da Encyclopedia of Life, ampliadas através de data augmentation (com filtros verticais e horizontais), distribuídas em conjuntos de treinamento (70\%), validação (20\%) e teste (10\%). O modelo foi treinado por 23 épocas com taxa de aprendizado de 0,00001 e função de ativação ReLU, empregando transfer learning em blocos convolucionais selecionados. A abordagem demonstrou alta acurácia e baixa taxa de erro nos três blocos de convolução mais eficientes. Em seus experimentos, realizados no município de Jacareí-SP, alcançaram uma acurácia superior a 90\% para as espécies Araucária e 99,9\% para as espécies de Pitanga.
No estudo de Carneiro (2023), desenvolveu-se um modelo de Deep Learning para identificação de Aspidosperma polyneuron (Peroba-Rosa) em fragmentos florestais, utilizando imagens de sensoriamento remoto integradas a técnicas de machine learning. Os dados de referência foram obtidos por meio de um inventário florestal prévio, que mapeou a localização exata de 302 indivíduos distribuídos em 126 hectares. Desse total, 100 indivíduos foram utilizados para treinamento do modelo e os demais para validação. Aplicou-se a técnica de data augmentation para ampliar artificialmente o conjunto de treinamento, gerando 1.162 amostras adicionais. O modelo empregado consistiu em uma Regional Convolutional Neural Network (R-CNN) da biblioteca ESRI, treinada com 10 épocas. Os resultados indicaram precisão de 23\% e acurácia de 71\% na avaliação de novos indivíduos, sugerindo a necessidade de aprimoramentos em pesquisas futuras para otimizar o desempenho do modelo.
Diante dos avanços promissores apresentados pelos estudos de de Souza et al. (2020) e Carneiro (2023), fica evidente a necessidade de ampliar pesquisas que explorem diferentes aspectos de treinamento de modelos de IA, como arquiteturas variadas, técnicas de aumento de dados e estratégias de transfer learning, visando superar os desafios de identificação de espécies em ambientes complexos. A otimização dessas ferramentas tecnológicas não só elevaria a precisão dos resultados, como também reforçaria seu impacto ecológico ao permitir o monitoramento eficiente de espécies ameaçadas e o planejamento de ações de conservação. Além disso, o aprimoramento desses sistemas traria significativos benefícios econômicos, viabilizando o manejo sustentável de recursos florestais valiosos e a geração de dados confiáveis para políticas públicas e iniciativas privadas.
