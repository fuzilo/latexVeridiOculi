As Redes Neurais Artificiais (RNAs) são sistemas inspirados no cérebro humano, desenvolvidos para criar máquinas capazes de aprender e realizar tarefas complexas. Pioneiros como \textcite{mcculloch1943logical} criaram o primeiro modelo de neurônio artificial, conhecido como LTU (Logical Threshold Unit), que executava funções lógicas simples, estabelecendo as bases para o desenvolvimento das RNAs. Essas redes simulam a capacidade de aprendizado do cérebro humano, dividindo-se em dois aspectos principais: a arquitetura, que define como as unidades de processamento estão conectadas, e o aprendizado, que ajusta os pesos das redes conforme novas informações são processadas.
Com o crescimento no volume de dados disponíveis e o avanço dos processadores, as RNAs evoluíram para redes profundas (Deep Networks - DNs), que têm superado outros algoritmos de aprendizado de máquina em tarefas como reconhecimento de imagens, voz e linguagem natural. Dentre essas redes, as Convolutional Neural Networks (CNNs) destacam-se pelo desempenho notável em reconhecimento de imagens, sendo inspiradas no processamento visual do cérebro para extrair padrões cada vez mais complexos.

Aplicações práticas dessas tecnologias são observadas em \textcite{de2020plantai}, um sistema desenvolvido para mapear plantas ameaçadas na Mata Atlântica. Utilizando CNNs, o aplicativo alcançou alta precisão na classificação de espécies, como Araucária e Pitanga, evidenciando o potencial das redes neurais na conservação ambiental. No entanto, nem todas as aplicações apresentam sucesso semelhante. Um exemplo é o modelo de \textcite{carneiro2023uso} , projetado para identificar espécies de Peroba-Rosa a partir de imagens de satélite, que apresentou baixa precisão, sugerindo que ainda há desafios a serem superados.

Em 2024, a Embrapa lançou o NetFlora ,\textcite{netflora}, e o MacView, \textcite{embrapa_agroenergia}, ferramentas que utilizam redes neurais para a identificação de espécies florestais. O NetFlora, focado no manejo sustentável da Amazônia, usa drones e inteligência artificial para identificar espécies com valor econômico, enquanto o MacView auxilia no mapeamento de palmeiras Macaúba e Babaçu, contribuindo para o desenvolvimento de planos de manejo sustentável. Esses avanços demonstram o contínuo progresso da aplicação de RNAs em áreas de importância ecológica e econômica.