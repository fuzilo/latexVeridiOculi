O sistema web proposto apresenta grande potencial ao utilizar inteligência artificial para análise automatizada de imagens aéreas em larga escala, substituindo métodos tradicionais que demandam alto custo e recursos técnicos especializados. Capaz de gerar laudos com acurácia superior a 80\%, o sistema oferece dados confiáveis para relatórios técnicos e científicos, além de otimizar processos decisórios no âmbito público. Um dos destaques é sua eficiência na identificação de embaúbas - espécie pioneira em áreas de regeneração -, cuja aglomeração em mapas cartesianos pode indicar estágios avançados de recuperação ambiental em locais previamente degradados. O aplicativo complementa essa abordagem ao permitir análises ágeis e de baixo custo, mesmo em casos com precisão reduzida. Ambas as soluções tornam-se aliadas estratégicas na identificação de espécies ameaçadas e no planejamento de restauração ecológica, democratizando o acesso a tecnologias que antes eram restritas a grandes projetos.