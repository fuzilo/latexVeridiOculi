A proposta do sistema web é promissora, pois utiliza inteligência artificial para analisar imagens aéreas em larga escala, automatizando processos que requerem recursos humanos e técnicos avançados.
Essa abordagem permite a geração de laudos com altas taxas de acurácia, que podem ser utilizados em relatórios técnicos científicos, e mesmo resultados menos precisos podem auxiliar na identificação de espécies em extinção. Além disso, o baixo custo dessas análises pode desempenhar um papel decisivo na tomada de decisões e auxiliar o poder público a automatizar processos de cálculos de Restauração de Flora.
O alto nível de acurácia obtido, acima de 80\% indica um bom desempenho do modelo na tarefa de classificação, refletindo a capacidade do modelo em generalizar novos dados.
A proposta do aplicativo é promissora, pois utiliza inteligência artificial para analisar imagens aéreas em larga escala, automatizando processos que requerem recursos humanos e técnicos avançados. Essa abordagem permite a geração de laudos com altas taxas de acurácia, que podem ser utilizados em relatórios técnicos científicos, e mesmo resultados menos precisos podem auxiliar na identificação de espécies em extinção. Além disso, o baixo custo dessas análises pode desempenhar um papel decisivo na tomada de decisões em projetos de recuperação da flora nativa e restauração de ambientes degradados por desmatamento.