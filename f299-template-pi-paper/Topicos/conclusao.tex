A proposta do sistema web é promissora, pois utiliza inteligência artificial para analisar imagens aéreas em larga escala, automatizando processos que tradicionalmente requerem recursos humanos e técnicos avançados. Essa abordagem permite a geração de laudos com altas taxas de acurácia, que podem ser utilizados em relatórios técnicos e científicos; mesmo resultados menos precisos podem auxiliar na identificação de espécies ameaçadas de extinção. Além disso, o baixo custo dessas análises pode desempenhar um papel decisivo na tomada de decisões e apoiar o poder público na automatização de processos de cálculo para a restauração da flora.

O alto nível de acurácia obtido, acima de 80\%, indica um bom desempenho do modelo na tarefa de classificação, refletindo sua capacidade de generalizar para novos dados.

A proposta do aplicativo é igualmente promissora, pois também utiliza inteligência artificial para a análise de imagens aéreas em larga escala, automatizando processos que requerem recursos humanos e técnicos avançados. Essa abordagem possibilita a geração de laudos com alta acurácia, aplicáveis em relatórios técnicos e científicos, e, mesmo quando os resultados apresentam menor precisão, podem ainda assim auxiliar na identificação de espécies ameaçadas de extinção. Ademais, o baixo custo dessas análises pode ter um papel crucial na tomada de decisões em projetos de recuperação da flora nativa e restauração de ambientes degradados pelo desmatamento.