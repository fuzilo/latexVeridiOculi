O objetivo do presente trabalho foi desenvolver uma aplicação baseada em redes neurais convolucionais (CNN) para identificação automatizada da Cecropia pachystachya (embaúba) em imagens aéreas, visando otimizar o processo de reconhecimento dessa espécie na Mata Atlântica e contribuir para a conservação da biodiversidade. Assim, o trabalho visa:
\
\begin{itemize}
\item Criar um sistema de identificação capaz de detectar a embaúba em larga escala, auxiliando no monitoramento de áreas de interesse ambiental.
\item Facilitar a elaboração de laudos de flora com maior precisão e eficiência.
\item Gerar relatórios automatizados, permitindo o rastreamento de espécies nativas e invasoras.
\item Disponibilizar relatórios dinâmicos para apoiar a tomada de decisões em manejo ambiental e políticas de preservação.
\end{itemize}





