O objetivo do presente trabalho é desenvolver um Sistema Web para identificação de flora por meio de imageamento aéreo, utilizando uma rede neural convolucional. Esse sistema tem por finalidade identificar árvores na região da Mata Atlântica. Através desse sistema, será possível auxiliar na elaboração de laudos de flora em projetos de empreendimentos imobiliários, visando a preservação e o equilíbrio da biodiversidade, além da identificação precisa e em larga escala de espécies invasoras. Assim, o trabalho visa otimizar o processo de identificação de espécies ameaçadas e contribuir para a conservação e preservação da biodiversidade da Mata Atlântica.
Nesta etapa do projeto, também será desenvolvida uma aplicação mobile para visualização dos relatórios de identificação das árvores. Esses relatórios serão gerados automaticamente por uma API, que utiliza uma inteligência artificial generativa para identificar e organizar os dados de inventário florestal previamente salvos em nuvem. Dessa forma, a aplicação mobile possibilitará o acesso rápido e eficiente aos relatórios de flora, apoiando a tomada de decisões e o monitoramento contínuo de áreas de interesse ambiental.\\
\
\begin{itemize}
\item Desenvolver um sistema mobile para visualização de relatórios;
\item Implementar a API em ambiente de nuvem;
\item Testar e validar a funcionalidade da IA generativa.
\end{itemize}