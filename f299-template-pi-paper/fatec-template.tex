%%%% fatec-article.tex, 2024/03/10

%% Classe de documento
\documentclass[
  a4paper,%% Tamanho de papel: a4paper, letterpaper (^), etc.
  12pt,%% Tamanho de fonte: 10pt (^), 11pt, 12pt, etc.
  english,%% Idioma secundário (penúltimo) (>)
  brazilian,%% Idioma primário (último) (>)
]{article}

%% Pacotes utilizados
\usepackage[]{fatec-article}
\Author{1}{Name={Derick França Justo\\Ricardo Keiti Kurita Matsumura\\Valmir Ribeiro Cardoso\\Vinicius da Silva Costa}}

\Author{2}{Name={\{derick.justo@fatec.sp.gov.br\}\\ \{ ricardo.matsumura@fatec.sp.gov.br\} \\ \{ valmir.ribeiro2@fatec.sp.gov.br\} \\ \{vinicius.costa59@fatec.sp.gov.br\}}}

%% Definição das palavras-chaves/keywords
\Keyword{1}{Inteligência Artificial}{Artificial Intelligence}
\Keyword{2}{Cecropia pachystachya}{Cecropia pachystachya}
\Keyword{3}{Mata Atlântica}{Atlantic Forest}

%%%% Resumo no idioma primário (brazilian)
\begin{Abstract}[brazilian]%% Idioma (brazilian ou english)
  A Mata Atlântica, um dos biomas mais ameaçados do Brasil, perdeu mais de 20.000 hectares entre 2020 e 2021, restando apenas 29,7\% de sua cobertura florestal original. Estudos indicam que 30\% das espécies arbóreas latino-americanas estão em risco de extinção, exigindo soluções inovadoras para monitoramento e preservação. Alinhado aos Objetivos de Desenvolvimento Sustentável (ODS) da ONU—especialmente ODS 15 (vida terrestre), ODS 13 (ação climática) e ODS 11 (cidades sustentáveis)—, este trabalho propõe um sistema automatizado para identificação de espécies-chave, como a Cecropia pachystachya (embaúba), árvore pioneira em áreas degradadas e de relevância medicinal. Desenvolveu-se uma rede neural convolucional (CNN) personalizada para detectar embaúbas em imagens aéreas, utilizando um dataset de 400 imagens (90 da espécie-alvo e 310 de controle). O modelo, treinado com validação cruzada (10 folds), alcançou 84,5\% de acurácia média, demonstrando robustez na classificação. A arquitetura incluiu três camadas convolucionais, max pooling, e funções de ativação ReLU/Softmax, otimizadas para evitar o overfitting. Complementarmente, uma aplicação móvel em React Native foi integrada a uma API em Flask (hospedada na AWS), permitindo a geração dinâmica de relatórios para projetos ambientais e imobiliários.
\end{Abstract}

%%%% Resumo no idioma secundário (english)
\begin{Abstract}[english]%% Idioma (brazilian ou english)
  The Atlantic Forest, one of Brazil’s most endangered biomes, lost over 20,000 hectares between 2020 and 2021, with only 29.7\% of its original forest cover remaining. Studies indicate that 30\% of Latin American tree species are at risk of extinction, demanding innovative solutions for monitoring and preservation. Aligned with the UN Sustainable Development Goals (SDGs)—particularly SDG 15 (life on land), SDG 13 (climate action), and SDG 11 (sustainable cities)—this study proposes an automated system to identify keystone species such as Cecropia pachystachya (embaúba), a pioneer tree in degraded areas which has medicinal relevance. A customized convolutional neural network (CNN) was developed to detect embaúba trees in aerial images, using a dataset of 400 images (90 of the target species and 310 controls). The model, trained with 10-fold cross-validation, achieved an average accuracy of 84.5\%, demonstrating robust classification performance. The architecture included three convolutional layers, max pooling, and ReLU/Softmax activation functions, optimized to prevent overfitting. Additionally, a React Native mobile app was integrated with a Flask API (hosted on AWS), enabling dynamic report generation for environmental and real estate projects.
\end{Abstract}

%% Processamento de entradas (itens) do índice remissivo (makeindex)
\makeindex%

%% Arquivo(s) de referências
\addbibresource{fatec-article.bib}

%% Início do documento
\begin{document}

% Seções e subseções
%\section{Título de Seção Primária}%

%\subsection{Título de Seção Secundária}%

%\subsubsection{Título de Seção Terciária}%

%\paragraph{Título de seção quaternária}%

%\subparagraph{Título de seção quinária}%

\section*{Introdução}%
\label{sect:intro}
Estudos indicam que a América Latina abriga 23.631 das 58.497 espécies de árvores existentes no mundo \cite{bgci}. Entretanto, dados recentes mostram que o desmatamento na Mata Atlântica brasileira atingiu 21.642 hectares entre 2020 e 2021 - área equivalente a mais de 28 mil campos de futebol, esse valor representa o maior índice registrado desde 2015 \cite{atlas-mata-atlantica}. Apenas 29,7\% da área total desse bioma ainda mantém cobertura florestal original, evidenciando seu elevado grau de degradação \cite{mapbiomas}. O desmatamento está diretamente associado à perda de biodiversidade vegetal, cerca de 30\% das espécies de árvores na América Latina estão ameaçadas de extinção, evidenciando os impactos da degradação ambiental. Diante desse cenário, a Organização das Nações Unidas (ONU) estabeleceu os 17 Objetivos de Desenvolvimento Sustentável (ODS), com metas globais para erradicar a pobreza, proteger o meio ambiente e combater as mudanças climáticas. Nesse contexto, destaca-se a ODS 15, que tem como objetivo proteger, restaurar e promover o uso sustentável dos ecossistemas terrestres, além de gerir as florestas de forma equilibrada, visando frear a perda de biodiversidade. A proteção de florestas nativas e ecossistemas impacta diretamente nas ações globais contra as mudanças climáticas, alinhando-se ao ODS 13 (Ação Contra a Mudança Global do Clima). Além disso, dados do MapBiomas revelam que 70\% da Mata Atlântica já sofreu alteração antrópica, o que reforça a urgência do ODS 11 (Cidades e Comunidades Sustentáveis), que visa promover assentamentos humanos mais seguros e ambientalmente equilibrados \cite{agenda2030}. 

No decorrer deste estudo foi realizada a identificação de uma espécie arbórea de nome científico \textit{Cecropia pachystachya}, popularmente conhecida como Embaúba, uma árvore perenifólia que atinge cerca de 25 metros de altura e 45 centímetros de diâmetro à Altura do Peito (DAP) em sua fase adulta. Seu tronco é reto, cilíndrico e fistuloso (oco por dentro). A casca externa é áspera e cinza-clara, com até 6mm de espessura, enquanto a casca interna é alaranjada-rosada e fibrosa. Suas folhas são alternadas e agrupadas nas extremidades dos ramos, com lâmina de 20 a 35 cm de comprimento por 20 a 35 cm de largura. Elas são palmatilobadas e divididas em 5 a 12 lobos desiguais obovados, separados até o pecíolo por espaços de 2 a 3 cm, e densamente esbranquiçadas-tomentosas (com penugem) na face inferior. A face superior apresenta pêlos curtos e esparsos, margem inteira ou ligeiramente ondulada e ápice obtuso, com nervura central proeminente na face inferior. O pecíolo é forte e comprido, medindo de 16 a 25 cm de comprimento, com pelos uncinados e cálix na base. É uma espécie dióica, ou seja, possui indivíduos machos e fêmeas. A Embaúba é uma espécie pioneira, associada a capoeiras novas, áreas de vegetação secundária que surgem em terrenos previamente desmatados, situadas junto a vertentes ou cursos d’água, estabelecendo-se rapidamente em clareiras descampadas. Popularmente é conhecida por seu uso medicinal da folha e casca. \cite{carvalho2006embauba} 

Em uma pesquisa de campo conduzida pelos autores, realizada por meio de questionários aplicados em 3 instituições (CETESB, Fundação Florestal e um escritório de Engenharia Ambiental), avaliou a adoção de tecnologias para preservação ambiental. Todos os entrevistados confirmaram compromisso com a causa, mas 78\% destacaram o alto custo de levantamentos de flora como uma barreira para garantir a preservação. Embora a maioria utilize tecnologias como drones, GIS e imagens aéreas, 44\% desconfiam de análises por Inteligência Artificial (IA), enquanto 56\% veem potencial como ferramenta auxiliar. A carência de recursos atualizados (como aerofotos recentes) eleva custos e limita inventários florestais, refletindo a divisão de opiniões sobre a viabilidade da IA na tomada de decisões.

Este trabalho visa desenvolver uma aplicação baseada em Inteligência Artificial (IA) para a identificação da Cecropia pachystachya (embaúba) por meio de imagens aéreas. A espécie, de relevância econômica e ambiental, desempenha um papel crucial na manutenção da biodiversidade. A ferramenta proposta tem como finalidade auxiliar na elaboração de inventários florestais e no levantamento detalhado de espécies arbóreas, servindo de suporte para projetos de manejo, conservação e empreendimentos imobiliários na Mata Atlântica. Além disso, busca fornecer embasamento técnico para propostas de rastreamento e preservação desta espécie emblemática.

\section*{OBJETIVO} \label{sect:obj}

O objetivo do presente trabalho é desenvolver um Sistema Web para identificação de flora por meio de imageamento aéreo, utilizando uma rede neural convolucional. Esse sistema tem por finalidade identificar árvores na região da Mata Atlântica. Através desse sistema, será possível auxiliar na elaboração de laudos de flora em projetos de empreendimentos imobiliários, visando a preservação e o equilíbrio da biodiversidade, além da identificação precisa e em larga escala de espécies invasoras. Assim, o trabalho visa otimizar o processo de identificação de espécies ameaçadas e contribuir para a conservação e preservação da biodiversidade da Mata Atlântica.
Nesta etapa do projeto, também será desenvolvida uma aplicação mobile para visualização dos relatórios de identificação das árvores. Esses relatórios serão gerados automaticamente por uma API, que utiliza uma inteligência artificial generativa para identificar e organizar os dados de inventário florestal previamente salvos em nuvem. Dessa forma, a aplicação mobile possibilitará o acesso rápido e eficiente aos relatórios de flora, apoiando a tomada de decisões e o monitoramento contínuo de áreas de interesse ambiental.\\
\
\begin{itemize}
\item Desenvolver um sistema mobile para visualização de relatórios;
\item Implementar a API em ambiente de nuvem;
\item Testar e validar a funcionalidade da IA generativa.
\end{itemize}

\section*{ESTADO DA ARTE} \label{sect:estadoarte}

A identificação de árvores de interesse ambiental e econômico torna-se um desafio em grandes áreas, especialmente devido à dificuldade de distinguir entre espécies com características semelhantes. A utilização de aprendizado profundo e Inteligência Artificial pode otimizar esse processo, tornando-o mais rápido, eficiente e com alta acurácia. Essa abordagem tecnológica possibilita a geração de dados confiáveis para subsidiar tomadas de decisão e a elaboração de estratégias de manejo sustentável.
Para classificar e mapear plantas ameaçadas na Mata Atlântica de forma colaborativa, de Souza et al. (2020) desenvolveram um aplicativo baseado em uma arquitetura MobileNet adaptada. Os autores utilizaram imagens da Encyclopedia of Life, ampliadas através de data augmentation (com filtros verticais e horizontais), distribuídas em conjuntos de treinamento (70\%), validação (20\%) e teste (10\%). O modelo foi treinado por 23 épocas com taxa de aprendizado de 0,00001 e função de ativação ReLU, empregando transfer learning em blocos convolucionais selecionados. A abordagem demonstrou alta acurácia e baixa taxa de erro nos três blocos de convolução mais eficientes. Em seus experimentos, realizados no município de Jacareí-SP, alcançaram uma acurácia superior a 90\% para as espécies Araucária e 99,9\% para as espécies de Pitanga.
No estudo de Carneiro (2023), desenvolveu-se um modelo de Deep Learning para identificação de Aspidosperma polyneuron (Peroba-Rosa) em fragmentos florestais, utilizando imagens de sensoriamento remoto integradas a técnicas de machine learning. Os dados de referência foram obtidos por meio de um inventário florestal prévio, que mapeou a localização exata de 302 indivíduos distribuídos em 126 hectares. Desse total, 100 indivíduos foram utilizados para treinamento do modelo e os demais para validação. Aplicou-se a técnica de data augmentation para ampliar artificialmente o conjunto de treinamento, gerando 1.162 amostras adicionais. O modelo empregado consistiu em uma Regional Convolutional Neural Network (R-CNN) da biblioteca ESRI, treinada com 10 épocas. Os resultados indicaram precisão de 23\% e acurácia de 71\% na avaliação de novos indivíduos, sugerindo a necessidade de aprimoramentos em pesquisas futuras para otimizar o desempenho do modelo.
Diante dos avanços promissores apresentados pelos estudos de de Souza et al. (2020) e Carneiro (2023), fica evidente a necessidade de ampliar pesquisas que explorem diferentes aspectos de treinamento de modelos de IA, como arquiteturas variadas, técnicas de aumento de dados e estratégias de transfer learning, visando superar os desafios de identificação de espécies em ambientes complexos. A otimização dessas ferramentas tecnológicas não só elevaria a precisão dos resultados, como também reforçaria seu impacto ecológico ao permitir o monitoramento eficiente de espécies ameaçadas e o planejamento de ações de conservação. Além disso, o aprimoramento desses sistemas traria significativos benefícios econômicos, viabilizando o manejo sustentável de recursos florestais valiosos e a geração de dados confiáveis para políticas públicas e iniciativas privadas.


\section*{METODOLOGIA} \label{sect:metodologia}

Este estudo propõe a identificação de árvores de embaúba, por meio de imagens, utilizando uma Convolutional Neural Networtk (CNN) para diferenciá-las de outras espécies arbóreas. A abordagem automatizada reduz a dependência de métodos manuais, dispendiosos e demorados, comumente empregados em inventários florestais. As Redes Neurais Artificiais (RNAs) são sistemas inspirados no cérebro humano, desenvolvidos para criar máquinas capazes de aprender e realizar tarefas complexas. Pioneiros como \cite{mcculloch1943logical} criaram o primeiro modelo de neurônio artificial, conhecido como Logical Threshold Unit (LTU), que executava funções lógicas simples, estabelecendo as bases para o desenvolvimento das RNAs. Essas redes simulam a capacidade de aprendizado do cérebro humano, dividindo-se em dois aspectos principais: a arquitetura, que define como as unidades de processamento estão conectadas, e o aprendizado, que ajusta os pesos das redes conforme novas informações são processadas.

Com o crescimento no volume de dados disponíveis e o avanço dos processadores, as RNAs evoluíram para redes profundas (Deep Networks - DN), que têm superado outros algoritmos de aprendizado de máquina em tarefas como reconhecimento de imagens, voz e linguagem natural. Dentre essas redes, as Redes Neurais Convolucionais (Convolutional Neural Networks - CNN) destacam-se pelo desempenho notável em reconhecimento de imagens, sendo inspiradas no processamento visual do cérebro para extrair padrões cada vez mais complexos. 

\subsection*{AMBIENTE DE TREINAMENTO}
Foi utilizado um hardware composto por um processador Intel(R) Core(TM) i7-9700K CPU 3.60GHz, 16 GB de memória RAM DDR4 e uma placa de vídeo NVIDIA GeForce GTX 960 com 4GB de memória DDR5. Essa configuração garantiu o treinamento dos modelos em tempo hábil, permitindo o ajuste das variáveis de entrada para o aperfeiçoamento do modelo.
Para a implementação do algoritmo de identificação de espécies arbóreas, foi construído um dataset contendo:
\begin{itemize}
    \item 90 imagens da classe embaúba(Cecropia spp.), coletadas em trabalho de campo pelos autores;
    \item 310 imagens de outras espécies arbóreas, obtidas de banco de imagens públicas, que serviram como classe de controle.
\end{itemize}

\subsection*{IMPLEMENTAÇÃO DO ALGORITMO TENSORFLOW}

Visando classificar automaticamente essas categorias, adotou-se uma CNN com arquitetura personalizada, composta por:

\begin{itemize}

    \item Três camadas convolucionais, com 32, 64 e 64 filtros (kernels 3x3), respectivamente;
    \item Camadas de MaxPooling (2x2) para redução dimensional;
    \item Funções de ativação ReLU (para não-linearidade) e Softmax (para classificação).
   
\end{itemize}

A - Camada de Entrada: Representada por um tensor bidimensional com dimensões 918x918 pixels de altura e largura.
B - 1ª Camada convolucional onde foram aplicados 32 filtros com kernel 3x3 a fim de extrair padrões visuais locais como bordas e texturas simples, resultando em um novo tensor, de mesmas dimensões, com 32 canais de ativação.


\begin{figure}[!h]
    \centering
    \caption{Evolução da acurácia ao longo dos treinamentos}
    \label{Gráfico 3}
    \includegraphics[width=0.9\linewidth]{Illustrations/tensorflow1.png}
\end{figure}

C - 1ª Camada MaxPooling em que aplicou-se pooling  com janelas 2x2, reduzindo a altura e largura pela metade. A redução das dimensões também auxilia a redução da complexidade computacional. O resultado é um tensor de 459x459 pixels e 32 canais de ativação.

D - 2ª Camada Convolucional. Nesta etapa aplicou-se 64 filtros 3x3, para detectar padrões mais complexos formados a partir das ativações da camada anterior.

E - 2ª Camada MaxPooling. Novamente reduziu-se as dimensões para reduzir a complexidade computacional.

F - 3ª Camada Convolucional, em que foram aplicados 64 filtros 3x3 para obter características ainda mais abstratas e específicas da imagem.
G - Camada Flatten(Achatamento). Converteu-se o tensor tridimensional em um vetor unidimensional totalizando 3.356.224 elementos.

H - Camada Densa com ReLU. Nesta camada totalmente conectada, recebeu o vetor e aplicou-se a função de ativação ReLU, que adiciona não-linearidade, que permite o aprendizado de combinações complexas de características.

\begin{figure}[!h]
    \centering
    \caption{Evolução da acurácia ao longo dos treinamentos}
    \label{Gráfico 3}
    \includegraphics[width=0.9\linewidth]{Illustrations/tensorflow2.png}
\end{figure}

I - Camada de Saída com \textit{SoftMax}. Nesta camada final, classifica-se a imagem em uma das categorias possíveis, convertendo valores em probabilidades, indicando a classe mais provável.

A escolha por uma CNN customizada - em vez de modelos pré-treinados como VGG ou ResNet - justifica-se pelo tamanho reduzido do dataset de 400 imagens, evitando assim o \textit{overfitting} (sobreajuste do modelo de dados de treinamento). O treinamento foi realizado do zero (\textit{from scratch}), sem transferência de aprendizado, para garantir a adaptação do modelo às características específicas das imagens coletadas. 
A fim de verificar como o modelo generaliza para dados novos, ou seja, como ele se comporta com dados que não foram usados durante o treinamento, utilizou-se a validação cruzada(\textit{K-Fold Cross Validation}). Neste método, os dados são divididos em K partes iguais, posteriormente ele é treinado K vezes, e a cada iteração é utilizado uma dessas partes como conjuntos de teste até completar todos. A métrica de desempenho é calculada 4 para cada uma das K execuções e ao final é feita a média desse valores.

O procedimento de avaliação do modelo envolve validação cruzada com 10 folds, visando garantir a robustez e a capacidade de generalização do modelo. O dataset é dividido em 10 partes, e o modelo é treinado e validado 10 vezes, utilizando, a cada rodada, uma parte distinta para validação enquanto as outras nove são usadas para treinamento.
O modelo é compilado com o otimizador Adam e treinado utilizando a função de perda \textit{Sparse Categorical Crossentropy}, apropriada para tarefas de classificação multiclasses com rótulos inteiros. A entropia cruzada calcula o erro comparando as probabilidades atribuídas pelo modelo a cada classe (saída) com as probabilidades reais (valores desejados), normalmente representadas por "1" para a classe correta e "0" para as demais. O objetivo é minimizar essa perda para que o modelo faça previsões mais precisas.

Com intuito de garantir que a rede neural seja confiável, seu aprendizado deve ser refinado gradualmente. Para tanto, o conjunto de dados é treinado várias vezes, denominando-se épocas, ou epoch em inglês. Durante o treinamento, são realizadas 100 épocas para assegurar que o modelo tenha tempo suficiente para aprender as características dos dados, com batch size definido como 32. A métrica de desempenho adotada é a acurácia, que mede a proporção de previsões corretas do modelo. Os valores de acurácia para cada fold são coletados, e a acurácia média é calculada para avaliar o desempenho geral do modelo.

\subsection*{IMPLEMENTAÇÃO DE APLICAÇÃO MOBILE PARA VISUALIZAÇÃO DE RELATÓRIOS}

Para o desenvolvimento de um sistema de visualização de relatórios de identificação de árvores em ambiente mobile, a aplicação será construída em React Native, proporcionando uma interface intuitiva e responsiva dada sua eficiência e versatilidade para criação de aplicações destinadas a múltiplas plataformas\cite{(ALMEIDA, 2024)}. Esta interface permitirá que os usuários naveguem pelos relatórios dentro da tela de análises (b). No relatório (d), são apresentados informações relacionadas à Embaúba e seus pontos cartesianos na imagem analisada. Além disso, será implementada uma função de atualização em tempo real, garantindo que as informações mais recentes processadas pela Interface de Programação de Aplicações (\textit{Application Programming Interface} - API) sejam refletidas nos relatórios. 

\begin{figure}[!h]
    \centering
    \caption{Evolução da acurácia ao longo dos treinamentos}
    \label{Gráfico 3}
    \includegraphics[width=0.8\linewidth]{Illustrations/aplicacaomobile.png}
  
\end{figure}

A API será desenvolvida em \textit{Flask} e hospedada em um ambiente de nuvem da Amazon Web Services (AWS), com dados armazenados em um \textit{bucket S3} . Esse setup em nuvem garantirá escalabilidade e alta disponibilidade, além de segurança no controle de acesso às informações. A API acessará e retornará os dados de inventário florestal conforme solicitado pela aplicação, após realizar testes de conectividade e otimizações de desempenho para melhor integração entre API e aplicação. Para validar a funcionalidade da inteligência artificial generativa utilizada na identificação das espécies arbóreas, serão realizados testes de precisão e eficiência, utilizando conjuntos de dados de teste e métodos de \textit{benchmarking}. A IA será avaliada quanto à sua capacidade de identificar corretamente espécies nativas e invasoras. Por fim, testes de integração garantirão que os relatórios gerados pela IA estejam formatados adequadamente para visualização no sistema mobile, assegurando a consistência e confiabilidade dos dados apresentados aos usuários.

\section*{RESULTADOS PRELIMINARES}\label{sect:resultados}

\subsection*{DESEMPENHO DO MODELO}
Resultados da Validação Cruzada
O desempenho do modelo foi avaliado utilizando validação cruzada (K-Fold Cross Validation) com 10 folds. A seguir estão os resultados de acurácia de validação para cada fold:

\begin{figure}[!h]
    \centering
    \caption{Resultado da Acurária ao longo dos treinamentos}
    \label{Gráfico 3}
    \includegraphics[width=0.7\linewidth]{Illustrations/tabela Acurácia.png}
\end{figure}

Análise da Acurácia Média A acurácia média de validação obtida ao longo dos 10 folds foi de 84.5\%, indicando um bom desempenho do modelo na tarefa de classificação. Esta métrica reflete a capacidade do modelo de generalizar bem para novos dados. 7 Variabilidade entre os Folds Houve alguma variação nos resultados de acurácia entre os diferentes folds, com um mínimo de 77.5\% e um máximo de 92.5\%.

\begin{figure}[!h]
    \centering
    \caption{Evolução da acurácia ao longo dos treinamentos}
    \label{Gráfico 3}
    \includegraphics[width=0.7\linewidth]{Illustrations/Figure_1.png}
    \SourceOrNote{Autoria Própria (2025)}
\end{figure}

Esta variabilidade é esperada devido a diferenças na distribuição dos dados em cada fold e sugere que, enquanto o modelo performa consistentemente bem, há espaço para melhorias na generalização.Figura 3 O modelo foi treinado por 100 épocas com um batch size de 32, usando o otimizador Adam e a função de perda sparse categorical crossentropy. Este número de épocas foi suficiente para permitir ao modelo aprender as características dos dados sem sobreajustar aos dados de treinamento. A função de ativação ReLU nas camadas densas e softmax na camada de saída ajudaram a capturar relações não-lineares e a classificar corretamente as imagens.

\section*{CONCLUSÃO}\label{sect:conclusao}

A proposta do sistema web é promissora, pois utiliza inteligência artificial para analisar imagens aéreas em larga escala, automatizando processos que requerem recursos humanos e técnicos avançados.
Essa abordagem permite a geração de laudos com altas taxas de acurácia, que podem ser utilizados em relatórios técnicos científicos, e mesmo resultados menos precisos podem auxiliar na identificação de espécies em extinção. Além disso, o baixo custo dessas análises pode desempenhar um papel decisivo na tomada de decisões e auxiliar o poder público a automatizar processos de cálculos de Restauração de Flora.
O alto nível de acurácia obtido, acima de 80\% indica um bom desempenho do modelo na tarefa de classificação, refletindo a capacidade do modelo em generalizar novos dados.
A proposta do aplicativo é promissora, pois utiliza inteligência artificial para analisar imagens aéreas em larga escala, automatizando processos que requerem recursos humanos e técnicos avançados. Essa abordagem permite a geração de laudos com altas taxas de acurácia, que podem ser utilizados em relatórios técnicos científicos, e mesmo resultados menos precisos podem auxiliar na identificação de espécies em extinção. Além disso, o baixo custo dessas análises pode desempenhar um papel decisivo na tomada de decisões em projetos de recuperação da flora nativa e restauração de ambientes degradados por desmatamento.

\printbibliography

%% Elementos pós-textuais (opcionais): Apêndice e Anexo
%Caso for utilizar, basta retirar o símbolo de % na frente do comando
%\input{./Extras/post-textual}

%% Fim do documento
\end{document}